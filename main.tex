\documentclass[landscape,11pt]{article}
\usepackage{csvsimple}
\usepackage{booktabs}
\usepackage[landscape,margin=0in]{geometry}
\begin{document}
\tiny

Duża skala niezbalansowania prowadzi to sprowadzenia klasy mniejszościowej to liczności, która nie pozwala na efektywne wykorzystanie bardziej złożonych metod, takich jak ADASYN czy SMOTE.

Duże niezbalansowanie powoduje też znaczny wzrost liczności komitetu, a więc poza jego ważeniem zaproponowana została redukcja do modelu dwuetapowego.
\newpage

\textbf{GNB}\\
\begin{tabular}{l||c|c|c||c|c|c|c|c||c|c|c|c|c||c|c|c|c|c||c|c|c|c|c}\toprule%
	\bfseries Dataset & \bfseries Reg & \bfseries OS & \bfseries US & 
	\multicolumn{10}{c||}{\bfseries Without OS} &
	\multicolumn{10}{c}{\bfseries With OS} \\
	
	& & & & 
	\multicolumn{5}{c||}{\bfseries Nonreduced} &
	\multicolumn{5}{c||}{\bfseries Reduced} &
	\multicolumn{5}{c||}{\bfseries Nonreduced} &
	\multicolumn{5}{c}{\bfseries Reduced} \\
	
	& & & &
	reg&wei&cwei&nwei&ncwei&
	reg&wei&cwei&nwei&ncwei&
	reg&wei&cwei&nwei&ncwei&
	reg&wei&cwei&nwei&ncwei
	\\\midrule
	
	\csvreader[head to column names]{results/GNB.csv}{}%
	{\dataset & \reg & \os & \us &
	
	\ereg & \ewei & \ecwei & \enwei & \encwei &
	\eregr & \eweir & \ecweir & \enweir & \encweir &
	\eregos & \eweios & \ecweios & \enweios & \encweios & 
	\eregros & \eweiros & \ecweiros & \enweiros & \encweiros
	
	\\}%
	\\\bottomrule	
\end{tabular}
\newpage

\textbf{kNN}\\
\begin{tabular}{l||c|c|c||c|c|c|c|c||c|c|c|c|c||c|c|c|c|c||c|c|c|c|c}\toprule%
	\bfseries Dataset & \bfseries Reg & \bfseries OS & \bfseries US & 
	\multicolumn{10}{c||}{\bfseries Without OS} &
	\multicolumn{10}{c}{\bfseries With OS} \\
	
	& & & & 
	\multicolumn{5}{c||}{\bfseries Nonreduced} &
	\multicolumn{5}{c||}{\bfseries Reduced} &
	\multicolumn{5}{c||}{\bfseries Nonreduced} &
	\multicolumn{5}{c}{\bfseries Reduced} \\
	
	& & & &
	reg&wei&cwei&nwei&ncwei&
	reg&wei&cwei&nwei&ncwei&
	reg&wei&cwei&nwei&ncwei&
	reg&wei&cwei&nwei&ncwei
	\\\midrule
	
	\csvreader[head to column names]{results/kNN.csv}{}%
	{\dataset & \reg & \os & \us &
	
	\ereg & \ewei & \ecwei & \enwei & \encwei &
	\eregr & \eweir & \ecweir & \enweir & \encweir &
	\eregos & \eweios & \ecweios & \enweios & \encweios & 
	\eregros & \eweiros & \ecweiros & \enweiros & \encweiros
	
	\\}%
	\\\bottomrule	
\end{tabular}
\newpage

\textbf{DTC}\\
\begin{tabular}{l||c|c|c||c|c|c|c|c||c|c|c|c|c||c|c|c|c|c||c|c|c|c|c}\toprule%
	\bfseries Dataset & \bfseries Reg & \bfseries OS & \bfseries US & 
	\multicolumn{10}{c||}{\bfseries Without OS} &
	\multicolumn{10}{c}{\bfseries With OS} \\
	
	& & & & 
	\multicolumn{5}{c||}{\bfseries Nonreduced} &
	\multicolumn{5}{c||}{\bfseries Reduced} &
	\multicolumn{5}{c||}{\bfseries Nonreduced} &
	\multicolumn{5}{c}{\bfseries Reduced} \\
	
	& & & &
	reg&wei&cwei&nwei&ncwei&
	reg&wei&cwei&nwei&ncwei&
	reg&wei&cwei&nwei&ncwei&
	reg&wei&cwei&nwei&ncwei
	\\\midrule
	
	\csvreader[head to column names]{results/DTC.csv}{}%
	{\dataset & \reg & \os & \us &
	
	\ereg & \ewei & \ecwei & \enwei & \encwei &
	\eregr & \eweir & \ecweir & \enweir & \encweir &
	\eregos & \eweios & \ecweios & \enweios & \encweios & 
	\eregros & \eweiros & \ecweiros & \enweiros & \encweiros
	
	\\}%
	\\\bottomrule	
\end{tabular}
    
        
\end{document}
